\documentclass[]{article}
\usepackage{lmodern}
\usepackage{amssymb,amsmath}
\usepackage{ifxetex,ifluatex}
\usepackage{fixltx2e} % provides \textsubscript
\ifnum 0\ifxetex 1\fi\ifluatex 1\fi=0 % if pdftex
  \usepackage[T1]{fontenc}
  \usepackage[utf8]{inputenc}
\else % if luatex or xelatex
  \ifxetex
    \usepackage{mathspec}
  \else
    \usepackage{fontspec}
  \fi
  \defaultfontfeatures{Ligatures=TeX,Scale=MatchLowercase}
\fi
% use upquote if available, for straight quotes in verbatim environments
\IfFileExists{upquote.sty}{\usepackage{upquote}}{}
% use microtype if available
\IfFileExists{microtype.sty}{%
\usepackage{microtype}
\UseMicrotypeSet[protrusion]{basicmath} % disable protrusion for tt fonts
}{}
\usepackage[margin=1in]{geometry}
\usepackage{hyperref}
\hypersetup{unicode=true,
            pdftitle={Sequential Stein Variational Gradient Descent for Time Series Model Estimation},
            pdfauthor={Gibson, Reich, and Ray in some order},
            pdfborder={0 0 0},
            breaklinks=true}
\urlstyle{same}  % don't use monospace font for urls
\usepackage{graphicx,grffile}
\makeatletter
\def\maxwidth{\ifdim\Gin@nat@width>\linewidth\linewidth\else\Gin@nat@width\fi}
\def\maxheight{\ifdim\Gin@nat@height>\textheight\textheight\else\Gin@nat@height\fi}
\makeatother
% Scale images if necessary, so that they will not overflow the page
% margins by default, and it is still possible to overwrite the defaults
% using explicit options in \includegraphics[width, height, ...]{}
\setkeys{Gin}{width=\maxwidth,height=\maxheight,keepaspectratio}
\IfFileExists{parskip.sty}{%
\usepackage{parskip}
}{% else
\setlength{\parindent}{0pt}
\setlength{\parskip}{6pt plus 2pt minus 1pt}
}
\setlength{\emergencystretch}{3em}  % prevent overfull lines
\providecommand{\tightlist}{%
  \setlength{\itemsep}{0pt}\setlength{\parskip}{0pt}}
\setcounter{secnumdepth}{0}
% Redefines (sub)paragraphs to behave more like sections
\ifx\paragraph\undefined\else
\let\oldparagraph\paragraph
\renewcommand{\paragraph}[1]{\oldparagraph{#1}\mbox{}}
\fi
\ifx\subparagraph\undefined\else
\let\oldsubparagraph\subparagraph
\renewcommand{\subparagraph}[1]{\oldsubparagraph{#1}\mbox{}}
\fi

%%% Use protect on footnotes to avoid problems with footnotes in titles
\let\rmarkdownfootnote\footnote%
\def\footnote{\protect\rmarkdownfootnote}

%%% Change title format to be more compact
\usepackage{titling}

% Create subtitle command for use in maketitle
\newcommand{\subtitle}[1]{
  \posttitle{
    \begin{center}\large#1\end{center}
    }
}

\setlength{\droptitle}{-2em}
  \title{Sequential Stein Variational Gradient Descent for Time Series Model
Estimation}
  \pretitle{\vspace{\droptitle}\centering\huge}
  \posttitle{\par}
  \author{Gibson, Reich, and Ray in some order}
  \preauthor{\centering\large\emph}
  \postauthor{\par}
  \predate{\centering\large\emph}
  \postdate{\par}
  \date{December 3, 2017}

\usepackage{multicol}

\begin{document}
\maketitle

\section{Introduction}\label{introduction}

Particle filtering suffers from limitations including:

\begin{itemize}
\item particle depletion
\item predict steps can be far from filtered steps?
\item high Monte Carlo variability
\end{itemize}

These can be addressed with some success by various strategies including
whatever it's called when you add new particles near current particles
if effective number of particles is too small.

Here we propose another approach that we hope will do better than
particle filtering. In this approach, Stein Variational Gradient Descent
(SVGD) is used to sequentially estimate the distribution of state
variables in each time step, conditional on observed data up through
that time. This method should overcome problems with particle depletion
and predictions that are far from the true states that come up with
particle filtering.

\section{Method Derivation}\label{method-derivation}

Let \(x_t\), \(t = 1, \ldots, T\) denote an unobserved state vector at
each time \(t\). For now, this state has to be continuous but we might
be able to discretize later?

Let \(y_t\), \(t = 1, \ldots, T\) denote an observed value at each time
\(t\).

The details about when we start observing the \(y_t\)'s relative to the
first \(x_t\) are unimportant.

For now, we're just going to write down the method to evaluate the
likelihood for a fixed set of parameters. This is not explicitly
Bayesian or frequentist.

But we could likely differentiate the approximation to the likelihood
derived here with respect to parameters \(\theta\) and plug that into
SVGD to estimate the posterior?

\subsection{Model Structure}\label{model-structure}

States:

\begin{itemize}
\item $X_1 \sim g_1(x_1 ; \xi)$
\item $X_t \vert X_{t-1} \sim g(x_t \vert x_{t - 1} ; \xi)$ for all $t = 2, \ldots, T$
\end{itemize}

Observations:

\begin{itemize}
\item $Y_t \vert X_{t} \sim h(y_t | x_t ; \zeta)$
\end{itemize}

Here, \(g_1(\cdot)\) and \(g(\cdot)\) are appropriately defined
probability density functions depending on parameters \(\xi\) and
\(h(\cdot)\) is an appropriately defined probability density function or
probability mass function depending on parameters \(\zeta\).

Define \(\theta = (\xi, \zeta)\) to be the full set of model parameters.

\subsection{Overview of SVGD}\label{overview-of-svgd}

SVGD can be used to estimate a (continuous only?) distribution (as a
mixture of normals? Is that right?). It requires as inputs a set of
initial values for centers of the normals (? or are they just
particles?) and a gradient of the density at a particular
particle/center point.

\subsection{Filtering}\label{filtering}

There are two types of filtering:

\begin{enumerate}
\def\labelenumi{\arabic{enumi}.}
\tightlist
\item
  sample of particles \(x_{1:T}^{(k)} \sim f(x_{1:T} | y_{1:T})\)
\item
  sample of particles \(x_{t}^{(k)} \sim f(x_{t} | y_{1:t})\) for each
  \(t = 1, \ldots, T\)
\end{enumerate}

Let's look at the second one. Assume we have a sample
\(x_{t-1}^{(k)} \sim f(x_{t-1} | y_{1:t-1})\)

\begin{align*}
p(x_{t} | y_{1:t}) &= \frac{f(x_t, y_t | y_{1:t-1})}{f(y_t | y_{1:t-1})} \\
 &\propto f(x_t, y_t | y_{1:t-1}) \\
 &= f(y_t | x_t) f(x_t | y_{1:t-1}) \\
 &= f(y_t | x_t) \int f(x_t, x_{t-1} | y_{1:t-1}) d x_{t-1} \\
 &= f(y_t | x_t) \int f(x_t | x_{t - 1}) f(x_{t-1} | y_{1:{t-1}}) dx_{t-1} \\
 &\approx f(y_t | x_t) \sum_{x_{t-1}^{(k)}} f(x_t | x_{t - 1}^{(k)})
\end{align*}

So \(\log\{p(x_{t} | y_{1:t})\}\) is approximately proportional to
\(\log\{f(y_t | x_t)\} + \log\{\sum_{x_{t-1}^{(k)}} f(x_t | x_{t - 1}^{(k)})\}\)

\subsection{Evaluating the Likelihood via
Filtering}\label{evaluating-the-likelihood-via-filtering}

Our goal (for now) is to evaluate the likelihood function

\begin{align*}
L(\theta \vert y_{1:T}) &= f(y_{1:T} ; \theta) \\
&= f(y_1 ; \theta) \prod_{t = 2}^T f(y_t \vert y_{1:t-1} ; \theta) \\
&= \int_{x_1} f(y_1, x_1 ; \theta) d x_1 \prod_{t = 2}^T \int_{x_t} f(y_t, x_t \vert y_{1:t-1} ; \zeta) d x_{t} \\
&= \int_{x_1} f(y_1 \vert x_1 ; \zeta) f(x_1 ; \xi) d x_1 \prod_{t = 2}^T \int_{x_t} f(y_t \vert x_t, y_{1:t-1} ; \zeta) f(x_t \vert y_{1:t-1} ;\xi) d x_t \\
&= \int_{x_1} f(y_1 \vert x_1 ; \zeta) f(x_1 ; \xi) d x_1 \prod_{t = 2}^T \int_{x_t} f(y_t \vert x_t ; \zeta) f(x_t \vert y_{1:t-1} ;\xi) d x_t \\
&\approx \sum_{x_1^{(k)}} f(y_1 \vert x_1^{(k)} ; \zeta) \prod_{t = 2}^T \sum_{x_{t|t-1}^{(k)}} f(y_t \vert x_{t|t-1}^{(k)} ; \zeta) \text{, where}
\end{align*}

\begin{align*}
x_1^{(k)} \sim f(x_1 ; \xi) \text{ and }
x_{t|t-1}^{(k)} \sim f(x_t \vert y_{1:t-1} ;\xi)
\end{align*}

Note that if we have a sample
\(x_{t-1|t-1}^{(k)} \sim f(x_{t-1} \vert y_{1:t-1} ;\xi)\), we can
obtain a sample \(x_{t|t-1}^{(k)} \sim f(x_t \vert y_{1:t-1} ;\xi)\)
from the transition density.

We will apply SVGD to iteratively obtain samples from the updated
distributions \(x_{t|t}^{(k)} \sim f(x_{t} \vert y_{1:t} ;\xi)\)
starting from samples
\(x_{t-1|t-1}^{(k)} \sim f(x_{t-1} \vert y_{1:t-1} ;\xi)\) at the
previous time step. To do this, we need to obtain the derivative of the
log of the density we want to estimate with respect to \(x_{t}\).

\begin{align*}
&\frac{d}{d x_t} \log\{f(x_{t} \vert y_{1:t}; \xi)\} = \frac{d}{d x_t} \log\left\{\frac{f(x_t \vert y_{1:t-1}) f(y_t \vert x_t, y_{1:t-1})}{f(y_t \vert y_{t:t-1})}\right\} \\
&\qquad = \frac{d}{d x_t} \left[ \log\left\{f(x_t \vert y_{1:t-1})\right\} + \log \left\{f(y_t \vert x_t)\right\} - \log\left\{f(y_t \vert y_{t:t-1})\right\} \right] \\
&\qquad = \frac{d}{d x_t} \log\left\{\int_{x_{t-1}}f(x_{t} \vert x_{t-1}, y_{1:t-1}; \xi)f(x_{t-1} \vert y_{1:t-1}; \xi) d x_{t-1} \right\} + \frac{d}{d x_t} \log \left\{ f(y_t \vert x_t) \right\} \\
&\qquad = \frac{\frac{d}{d x_t} \int_{x_{t-1}}f(x_{t} \vert x_{t-1}; \xi)f(x_{t-1} \vert y_{1:t-1}; \xi) d x_{t-1}}{\int_{x_{t-1}}f(x_{t} \vert x_{t-1}; \xi)f(x_{t-1} \vert y_{1:t-1}; \xi) d x_{t-1}} + \frac{\frac{d}{d x_t} f(y_t \vert x_t)}{f(y_t \vert x_t)} \\
&\qquad \approx \frac{\frac{d}{d x_t} \sum_{x_{t-1|t-1}^{(k)}}f(x_{t} \vert x_{t-1}; \xi)}{\sum_{x_{t-1|t-1}^{(k)}}f(x_{t} \vert x_{t-1}; \xi)} + \frac{\frac{d}{d x_t} f(y_t \vert x_t)}{f(y_t \vert x_t)} \\
&\qquad = \frac{\sum_{x_{t-1|t-1}^{(k)}} \frac{d}{d x_t} f(x_{t} \vert x_{t-1}; \xi)}{\sum_{x_{t-1|t-1}^{(k)}}f(x_{t} \vert x_{t-1}; \xi)} + \frac{\frac{d}{d x_t} f(y_t \vert x_t)}{f(y_t \vert x_t)}
\end{align*}

\subsection{Simulation Studies}\label{simulation-studies}

We will do several simulation studies, divided into 2 groups:

\begin{enumerate}
\def\labelenumi{\arabic{enumi}.}
\tightlist
\item
  illustrating scenarios in which common particle filtering methods
  struggle, but SSVGD has better chances

  \begin{enumerate}
  \def\labelenumii{\alph{enumii}.}
  \tightlist
  \item
    bad initalization
  \item
    normal-poisson filtering, seasonal model
  \item
    one other, more complex?
  \end{enumerate}
\item
  demonstrating accuracy (compare to true states, exactly computed
  filtered states, and likelihood)

  \begin{enumerate}
  \def\labelenumii{\alph{enumii}.}
  \tightlist
  \item
    Kalman filter
  \item
    something nonlinear where we can do exact computations (brute force
    for short time series??)?
  \end{enumerate}
\end{enumerate}

\subsubsection{1. a. Bad Initialization}\label{a.-bad-initialization}

\paragraph{Simulation Study Design}\label{simulation-study-design}

Paragraph with data generating process, written with math.

Paragraph describing settings for simulation, e.g.~number of simulation
runs, length of time series generated, etc.

Paragraph describing different methods in comparison. 2 PF
implementations, one non-linear KF implementation, and SSVGD

\paragraph{Results}\label{results}

Paragraph, referencing one figure and one table, summarizing results

\subsubsection{1. b. Normal-Poisson seasonal
model}\label{b.-normal-poisson-seasonal-model}

\paragraph{Simulation Study Design}\label{simulation-study-design-1}

Paragraph with data generating process, written with math.

Paragraph describing settings for simulation, e.g.~number of simulation
runs, length of time series generated, etc.

Paragraph describing different methods in comparison. 2 PF
implementations, one non-linear KF implementation, and SSVGD

\paragraph{Results}\label{results-1}

Paragraph, referencing one figure and one table, summarizing results

\subsubsection{1. c. One other, more
complex?}\label{c.-one-other-more-complex}

\paragraph{Simulation Study Design}\label{simulation-study-design-2}

Paragraph with data generating process, written with math.

Paragraph describing settings for simulation, e.g.~number of simulation
runs, length of time series generated, etc.

Paragraph describing different methods in comparison. 2 PF
implementations, one non-linear KF implementation, and SSVGD

\paragraph{Results}\label{results-2}

Paragraph, referencing one figure and one table, summarizing results

\subsubsection{2. a. Kalman Filter}\label{a.-kalman-filter}

\paragraph{Simulation Study Design}\label{simulation-study-design-3}

Paragraph with data generating process, written with math.

Paragraph describing settings for simulation, e.g.~number of simulation
runs, length of time series generated, etc.

Paragraph describing different methods in comparison. 2 PF
implementations, one non-linear KF implementation, and SSVGD

\paragraph{Results}\label{results-3}

Paragraph, referencing one figure and one table, summarizing results

\subsubsection{2. b. something nonlinear where we can do exact
computations (brute force for short time
series??)?}\label{b.-something-nonlinear-where-we-can-do-exact-computations-brute-force-for-short-time-series}

\paragraph{Simulation Study Design}\label{simulation-study-design-4}

Paragraph with data generating process, written with math.

Paragraph describing settings for simulation, e.g.~number of simulation
runs, length of time series generated, etc.

Paragraph describing different methods in comparison. 2 PF
implementations, one non-linear KF implementation, and SSVGD

\paragraph{Results}\label{results-4}

Paragraph, referencing one figure and one table, summarizing results

\subsection{Application}\label{application}

Example model with real data. fairly real model, but not thaaaaaat
complex.

\subsection{Discussion}\label{discussion}


\end{document}
